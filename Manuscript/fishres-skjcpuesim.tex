\documentclass[authoryear,preprint,review,12pt]{elsarticle}

\usepackage{lineno,hyperref,amssymb}
\modulolinenumbers[5]

\journal{Fisheries Research}

%%%%%%%%%%%%%%%%%%%%%%%
%% Elsevier bibliography styles
%%%%%%%%%%%%%%%%%%%%%%%
%% APA style
\bibliographystyle{apa}
\biboptions{authoryear}

%%%%%%%%%%%%%%%%%%%%%%%

\begin{document}

\begin{frontmatter}

\title{Impacts of fisheries-dependent spatial sampling patterns on index standardization: A simulation study and fishery application}


%% Authors include affiliations in footnotes:
\author[address.1]{Nicholas Ducharme-Barth\corref{mycorrespondingauthor}}
\cortext[mycorrespondingauthor]{Corresponding author}
\ead{nicholasd@spc.int}
\author[address.2]{Arnaud Gr\"uss}
\author[address.3]{Junji Kinoshita}
\author[address.4]{Yoshinori Aoki}
\author[address.4]{Hidetada Kiyofuji}
\author[address.1]{Matthew Vincent}
\author[address.1]{John Hampton}
\author[address.1]{Graham Pilling}

\address[address.1]{Pacific Community, B.P. D5 98848 Noumea, New Caledonia}
\address[address.2]{School of Aquatic and Fishery Sciences, University of Washington, Box 355020, Seattle, WA, 98105-5020, USA}
\address[address.3]{National Research Institute of Far Seas Fisheries, Japan Fisheries Research and Education Agency, Yokohama, Kanagawa, Japan}
\address[address.4]{National Research Institute of Far Seas Fisheries, Japan Fisheries Research and Education Agency, Shimizu-ku, Shizuoka-shi, Shizuoka 424-8633 Japan}

\begin{abstract}
Abundance indices derived from fisheries dependent data (catch-per-unit-effort or CPUE) have a known potential for bias. These biases can arise from gear effects (saturation of the gear), systemic and structural changes to the fishing fleet over time (effort creep), and/or from non-random sampling relative to the spatiotemporal distribution of the underlying fish population. However, given the cost and lack of availability of fisheries independent surveys, these fisheries dependent CPUE remain a common and informative input to stock assessments. Given their common use, increasingly sophisticated standardization methods have been developed in order to standardize CPUE to remove the effects of gear, vessel, and spatial sampling. Recent research efforts have focused on the development of spatiotemporal models which simultaneously standardize the CPUE and interpolate abundance into unfished areas when estimating the index. These spatiotemporal models can be aided by environmental covariates (e.g. sst) and indices (e.g. ENSO) to interpolate into unfished areas. These spatiotemporal models have been demonstrated in simulation studies to perform better than conventional delta-generalized linear models. However, they have not been evaluated in situations where the spatial sampling coverage changes over time (e.g. fisheries expansion or spatial closures). This paper develops a simulation framework to evaluate 1) how the nature of spatial fisheries dependent sampling patterns may bias estimated abundance indices, 2) how temporal shifts in spatial sampling impact our ability to estimate temporal changes in catchability, and 3) how including an environmental covariate and/or a spatially varying coefficient in the formulation of the spatiotemporal model can improve the estimation of abundance indices given these shifts in spatial sampling. These models are then applied to a case study example where the spatial sampling pattern changed dramatically over time (contraction of the Japanese pole-and-line fishery for skipjack tuna in the western and central Pacific). The results indicate that the dramatic shifts in the spatial sampling pattern result in biased estimated indices though the inclusion of environmental covariates can mediate the magnitude of the bias and error in certain scenarios. Furthermore, these shifts in spatial sampling prevent the model from disentangling changes in abundance from changes in catchability.

      
\end{abstract}

\begin{keyword}
Spatial sampling\sep spatiotemporal models\sep CPUE \sep fisheries dependent data
\end{keyword}

\end{frontmatter}

\linenumbers

\section{Introduction}\label{Intro}
Abundance indices derived from fisheries dependent data remain a common and informative input to stock assessment models despite the known potential for bias. These biases can arise from gear effects (saturation of the gear,\citet{deriso_odds_1987}), systemic and structural changes to the fishing fleet over time (effort creep, \citet{bishop_analysing_2004,ye_how_2009}), and/or from non-random sampling relative to the spatiotemporal distribution of the underlying fish population \citep{clark_aggregation_1979, rose_effects_1991, rose_hyperaggregation_1999,swain_fish_1994}. 

Differences in gear configuration and fishing power Nominal fisheries catch-per-unit-effort (CPUE) trends can deviate   

Hyperstability/depletion...
Cost and lack of availability of fisheries independent surveys mean that they are still used.

Given their common use, a lot of research has been done to develop increasingly sophisticated standardization methods. appropriately standardize these indices to remove the effects of gear, vessel, and spatial sampling. Overview of methods involved.

Focus on spatiotemporal models and comparisons with existing methods...

While fisheries independent data come from statistically designed surveys that ensure the random distribution of samples across the spatial domain and temporal strata, the same assumption of appropriate spatiotemporal coverage cannot be made for fisheries dependent data. Holes in the spatiotemporal coverage from fisheries dependent data can arise from sampling preferentially with respect to abundance, changes in spatial targeting due to economic or management factors, as well as restricted access to fishing grounds due to regulatory or competitive forces. These anomalies in spatiotemporal sampling could lead to a disconnect between the underlying species abundance trend and the trend estimated from catch rate data, thus producing a biased index. Beyond the fisheries dependent simulation testing already conducted \citep{gruss_evaluation_2019,zhou_catch_2019}, there exists a need to test these spatiotemporal methods in the case where fisheries spatial sampling coverage changes over time.  

\section{Methods}\label{Methods}

\section{Results}\label{Results}

\section{Discussion}\label{Discussion}

\section{Acknowledgments}\label{Ack}

\bibliography{C:/Users/nicholasd/HOME/SPC/SPC_Misc/References/zotero}

\end{document}