\documentclass[authoryear,preprint,review,12pt]{elsarticle}

\usepackage{lineno,hyperref,amssymb}
\modulolinenumbers[5]

\journal{Fisheries Research}

%%%%%%%%%%%%%%%%%%%%%%%
%% Elsevier bibliography styles
%%%%%%%%%%%%%%%%%%%%%%%
%% APA style
\bibliographystyle{apa}
\biboptions{authoryear}

%%%%%%%%%%%%%%%%%%%%%%%

\begin{document}

\begin{frontmatter}

\title{Impacts of fisheries-dependent spatial sampling patterns on index standardization: A simulation study and fishery application}


%% Authors include affiliations in footnotes:
\author[address.1]{Nicholas Ducharme-Barth\corref{mycorrespondingauthor}}
\cortext[mycorrespondingauthor]{Corresponding author}
\ead{nicholasd@spc.int}
\author[address.2]{Arnaud Gr\"uss}
\author[address.3]{Junji Kinoshita}
\author[address.4]{Yoshinori Aoki}
\author[address.4]{Hidetada Kiyofuji}
\author[address.1]{Matthew Vincent}
\author[address.1]{John Hampton}
\author[address.1]{Graham Pilling}

\address[address.1]{Pacific Community, B.P. D5 98848 Noumea, New Caledonia}
\address[address.2]{School of Aquatic and Fishery Sciences, University of Washington, Box 355020, Seattle, WA, 98105-5020, USA}
\address[address.3]{National Research Institute of Far Seas Fisheries, Japan Fisheries Research and Education Agency, Yokohama, Kanagawa, Japan}
\address[address.4]{National Research Institute of Far Seas Fisheries, Japan Fisheries Research and Education Agency, Shimizu-ku, Shizuoka-shi, Shizuoka 424-8633 Japan}

\begin{abstract}
Blah blah blah...
\end{abstract}

\begin{keyword}
Spatial sampling\sep spatiotemporal models\sep CPUE \sep fisheries dependent data
\end{keyword}

\end{frontmatter}

\linenumbers

\section{Introduction}\label{Intro}
Abundance indices derived from fisheries dependent data remain a common and informative input to stock assessment models despite the known potential for bias. These biases can arise from gear effects (saturation of the gear,\citet{deriso_odds_1987}), systemic and structural changes to the fishing fleet over time (effort creep, \citet{bishop_analysing_2004,ye_how_2009}), and/or from non-random sampling relative to the spatiotemporal distribution of the underlying fish population \citep{clark_aggregation_1979, rose_effects_1991, rose_hyperaggregation_1999,swain_fish_1994}. 

Differences in gear configuration and fishing power Nominal fisheries catch-per-unit-effort (CPUE) trends can deviate   

Hyperstability/depletion...
Cost and lack of availability of fisheries independent surveys mean that they are still used.

Given their common use, a lot of research has been done to develop increasingly sophisticated standardization methods. appropriately standardize these indices to remove the effects of gear, vessel, and spatial sampling. Overview of methods involved.

Focus on spatiotemporal models and comparisons with existing methods...

While fisheries independent data come from statistically designed surveys that ensure the random distribution of samples across the spatial domain and temporal strata, the same assumption of appropriate spatiotemporal coverage cannot be made for fisheries dependent data. Holes in the spatiotemporal coverage from fisheries dependent data can arise from sampling preferentially with respect to abundance, changes in spatial targeting due to economic or management factors, as well as restricted access to fishing grounds due to regulatory or competitive forces. These anomalies in spatiotemporal sampling could lead to a disconnect between the underlying species abundance trend and the trend estimated from catch rate data, thus producing a biased index. Beyond the fisheries dependent simulation testing already conducted \citep{gruss_evaluation_2019,zhou_catch_2019}, there exists a need to test these spatiotemporal methods in the case where fisheries spatial sampling coverage changes over time.  

\section{Methods}\label{Methods}

\section{Results}\label{Results}

\section{Discussion}\label{Discussion}

\section{Acknowledgments}\label{Ack}

\bibliography{C:/Users/nicholasd/HOME/SPC/SPC_Misc/References/zotero}

\end{document}